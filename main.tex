%-------------------------
% Nick's Resume
%
% Licensed from Aras Gungore, who has released it under
% MIT license.
%------------------------

\documentclass[letterpaper,12pt]{article}

\usepackage{fontspec}
\usepackage{graphicx}
\usepackage{latexsym}
\usepackage[empty]{fullpage}
\usepackage{titlesec}
\usepackage[usenames,dvipsnames]{color}
\usepackage{verbatim}
\usepackage{enumitem}
\usepackage[hidelinks]{hyperref}
\usepackage{fancyhdr}
\usepackage[english]{babel}
\usepackage{tabularx}
\usepackage{fontawesome5}
\usepackage{academicons}
\usepackage{changepage}

%---------- DOCUMENT SETTINGS -------------

% Fix for the font awesome package from
% https://tex.stackexchange.com/questions/257574/fontawesome-doesnt-scale-up
\DeclareFontFamily{U}{fontawesome1}{}
\DeclareFontShape{U}{fontawesome1}{m}{n}{<->FontAwesome--fontawesomeone}{}
\DeclareFontFamily{U}{fontawesome2}{}
\DeclareFontShape{U}{fontawesome2}{m}{n}{<->FontAwesome--fontawesometwo}{}
\DeclareFontFamily{U}{fontawesome3}{}
\DeclareFontShape{U}{fontawesome3}{m}{n}{<->FontAwesome--fontawesomethree}{}
\DeclareRobustCommand{\FAone}{\usefont{U}{fontawesome1}{m}{n}}
\DeclareRobustCommand{\FAtwo}{\usefont{U}{fontawesome2}{m}{n}}
\DeclareRobustCommand{\FAthree}{\usefont{U}{fontawesome3}{m}{n}}

% Font settings
\newfontfamily{\titlefont}{Iosevka Bold}
\newfontfamily{\headingfont}{Iosevka Slab}
\setmainfont{Baskerville}

% Faux/Fake Small Caps from
% https://tex.stackexchange.com/questions/55664/fake-small-caps-with-xetex-fontspec
\makeatletter
\newlength\fake@f
\newlength\fake@c
\def\fakesc#1{%
  \begingroup%
  \xdef\fake@name{\csname\curr@fontshape/\f@size\endcsname}%
  \fontsize{\fontdimen8\fake@name}{\baselineskip}\selectfont%
  \uppercase{#1}%
  \endgroup%
}
\makeatother
\newcommand\fauxsc[1]{\fauxschelper#1 \relax\relax}
\def\fauxschelper#1 #2\relax{%
  \fauxschelphelp#1\relax\relax%
  \if\relax#2\relax\else\ \fauxschelper#2\relax\fi%
}
\def\Hscale{.93}\def\Vscale{.72}\def\Cscale{1.00}
\def\fauxschelphelp#1#2\relax{%
  \ifnum`#1=\lccode`#1\relax\scalebox{\Hscale}[\Vscale]{\char\uccode`#1}\else%
    \scalebox{\Cscale}[1]{#1}\fi%
  \ifx\relax#2\relax\else\fauxschelphelp#2\relax\fi}

% Header and Footer reset
\pagestyle{fancy}
\fancyhf{}
\fancyfoot{}
\renewcommand{\headrulewidth}{0pt}
\renewcommand{\footrulewidth}{0pt}

% Adjust margins
\addtolength{\oddsidemargin}{-0.5in}
\addtolength{\evensidemargin}{-0.5in}
\addtolength{\textwidth}{1in}
\addtolength{\topmargin}{-.5in}
\addtolength{\textheight}{1.0in}

% Set URL style to match document
\urlstyle{same}

\raggedbottom
\raggedright
\setlength{\tabcolsep}{0in}

% Section Header formatting
\titleformat{\section}{
  \vspace{-4pt}\headingfont\raggedright\large
}{}{0em}{}[\color{black}\titlerule \vspace{-5pt}]

%---------- CUSTOM COMMANDS --------------
% Commands set up commonly completed operations or blocks

% Creates a heading with an item on the left and an item on the right of the page
\newcommand{\simpleHeading}[2]{
    \vspace{-1pt}
    \begin{tabular*}{0.99\textwidth}[t]{l@{\extracolsep{\fill}}r}
        #1 & #2 \\
    \end{tabular*}
}

% Used for a major degree
\newcommand{\majorDegreeSubHeading}[2]{
    \simpleHeading{\hspace{7pt}#1}{\textit{#2}}
}

% Used for a minor degree
\newcommand{\minorDegreeSubHeading}[2]{
    \simpleHeading{\hspace{14pt}#1}{\textit{#2}}
}

% Used for a job position. Fields are: {Company}{Location}{Position}{Date}
\newcommand{\positionHeading}[4]{
    \simpleHeading{\textbf{#1}}{\textit{#2}}
    \simpleHeading{\textit{#3}}{\textit{#4}}
}

% Defines a simple paragraph for positions or projects
\newcommand{\simpleParagraph}[1]{
    \vspace{-1pt}
    \begin{adjustwidth}{7pt}{0pt}
        #1
    \end{adjustwidth}
}

\newcommand{\location}[0]{
    {\scriptsize\faIcon{map-marker}}
}

%---------- LINKS -----------

\newcommand{\matlab}[0]{
    Matlab
}

%---------- RESUME -----------

\begin{document}

%---------- HEADING ----------

{\Huge \titlefont{\fauxsc{Nick Waddoups}}} \\
\vspace{3pt}
\small

% '|' separated version
\faIcon{envelope} \hspace{.5pt} \href{mailto:npw1010@gmail.com}{npw1010@gmail.com}
\hspace{0.0625in}$|$\hspace{0.0625in}
\faIcon{mobile-alt} \hspace{.5pt} \href{tel:13853993215}{(385) 399-3215}
\hspace{0.0625in}$|$\hspace{0.0625in}
\faIcon{github} \hspace{.5pt} \href{https://github.com/nwad123}{github.com/nwad123}
%\hspace{0.0625in}$|$\hspace{0.0625in}
%\faIcon{globe} \hspace{.5pt} \href{https://nwad123.github.io/assume_false}{assumefalse.com}

%----------- EDUCATION -----------

\section{Education}
      \simpleHeading{\textbf{Utah State University}}{\textit{Logan, UT}}
      \majorDegreeSubHeading{M.Sc. in Computer Engineering. \textit{GPA: 4.00/4.00} }{2024 -- 2025}
      \majorDegreeSubHeading{B.Sc. in Computer Engineering.}{2020 -- 2024}
      \minorDegreeSubHeading{-- Minor in Anticipatory Intelligence.}{}
      \vspace{-15pt}

%----------- EXPERIENCE ------------

\section{Experience}

%----------- WORK EXPERIENCE -----------

    %% NG Experience
    \positionHeading{Northrop Grumman}{Utah}{Electrical Engineering Intern}{2022 -- 2024}

    \simpleParagraph{
        \begin{itemize}
            \item Wrote a facility electromagnetic pulse (EMP) shielding effectiveness test plan in
                accordance with MIL-STD-188-125-1. Test plan defined test procedures, setup, exposure
                limits, and required shielding effectiveness to demonstrate the facility meets customer 
                EMP hardness requirements.

            %\item Developed a \matlab script that calculated planar shielding effectiveness of any
            %    material across any frequency range. This \matlab script enabled EM designers to understand
            %    and estimate shielding characteristics without having to perform physical tests or expensive 
            %    simulations, reducing facility shielding development cost and schedule.
            %
            %\item Developed a Python script to characterize radiation-based Single Event Effects (SEE) in
            %    natural space. The script processed 1,000,000+ datapoints and reduced SEE analysis
            %    run times from several hours to 30 seconds.

            \item Designed and developed a robust \matlab GUI application that processed hundreds
                of test-data files and generated print-ready figures and tables showing test trends and outcomes.
                The \matlab script reduced data processing time by 90\% compared to the previous analysis 
                method in Excel.

            %\item Developed a MATLAB script that automated the process of building cable cross sections
            %    in Dassault CST Studio from cable procurement drawings, reducing CST-model setup time by 75\%.
            %    These cable cross sections are used to perform environmental EMI/EMC analysis for the 
            %    customer.

            \item Developed a MATLAB script that used signal processing techniques to characterize
                shielding effectiveness affects on time-domain transients by providing an estimated 
                attenuated time-domain transient after the original transient had passed through the shield.


        \end{itemize}
    }

    %% Research Experience
    %\vspace{10pt}
    %\positionHeading{Utah State University}{}{Formal Verification Researcher}{2022 -- Present (part time)}

    %% TA Experience
    %\vspace{10pt}
    %\positionHeading{Utah State University}{}
    %    {Computer Architecture and Digital Design TA}
    %    {2022 -- Present (part time)}

%----------- RELEVANT COURSEWORK -----------

    %% Computer Architecture Courses
    \vspace{10pt}
    \simpleHeading{\textbf{Computer Architecture Coursework}}
                  {\textit{USU College of Electrical and Computer Engineering}}

    \simpleParagraph{
        \begin{itemize}
            \item Developed three L2 cache prefetchers in C++ within the trace-based ChampSim CPU
                simulator framework. The simulation showed the performance difference between a next-line, stride,
                and best-offset prefetcher in terms of IPC and L2 miss rate.

            %\item Developed a configurable, trace-based CPU memory-cache simulator in C. The simulator
            %    read in memory traces generated by \textit{valgrind} and simulated the behavior of a single 
            %    level cache. Configuration of cache size was done via command line arguments, and the default
            %    replacement least-recently-used replacement policy could easily be switched out by editing the 
            %    source code.

            \item Designed and administered a cache-based side-channel communication lab. The lab taught
                students how to use CPU timers on an Intel x86 system to measure and exploit memory access 
                latency to illegally send information from one process to another using a prime and probe 
                technique in the L2 and L3 caches.

        \end{itemize}
    }

    %% VLSI Courses
    %\vspace{10pt}
    %\simpleHeading{\textbf{VLSI Coursework}}{\textit{USU College of Electrical and Computer Engineering}}
    %
    %\simpleParagraph{
    %    \begin{itemize}
    %        \item Designed and implemented a 3D global maze router based on an A* algorithm using C++17. The algorithm
    %            built on a pre-existing codebase that handled file IO and provided some basic functionality.
    %
    %        \item Wrote a command line application that used simulated annealing and slicing
    %            trees to generate optimal cell placement solutions for given given input cells.
    %
    %    \end{itemize}
    %}
    
    %% Reconfigurable / FPGA Courses
    %\vspace{10pt}
    %\simpleHeading{\textbf{FPGA Design Courses}}{\textit{USU College of Electrical and Computer Engineering}}

    %\simpleParagraph{
    %    \begin{itemize}
    %        \item 5-stage processor
    %        \item Brick-breaker game
    %    \end{itemize}
    %}
    
    %% Formal Verification Course
    \vspace{10pt}
    \simpleHeading{\textbf{Formal Verification Course and Research}}{\textit{USU College of Electrical and Computer Engineering}}

    \simpleParagraph{
        \begin{itemize}
            \item Verified a Network-on-Chip formal model using the M\fauxsc{odest} Toolset and used probabilitic and 
                statistical model checking to demonstrate how power supply noise in a Network-on-Chip is effected by 
                the Network-on-Chip design.

            \item Learned to use the Dafny program language and deductive reasoning to write correct software.
        \end{itemize}
    }
    
    %% Compilers / CS Courses (include assembler from reconfigurable and computer arch perhaps?)
    \vspace{10pt}
    \simpleHeading{\textbf{Compiler Coursework}}{\textit{USU College of Electrical and Computer Engineering}}

    \simpleParagraph{
        \begin{itemize}
            \item Wrote an assembler using C++ for a custom FPGA soft processor based on the DLX ISA (similar
                to RISC-V). The assembler parsed custom DLX assembly code, performed a pass to resolve
                loop references, and then emitted machine code in a format that could be loaded into the FPGA's 
                read-only memory section.

            \item Devloped a compiler using Java that compiled the toy "B++" language. The compiler supported basic 
                types, loops, function calls, and arithmetic operations and produced MIPS assembly that
                was run using the MARS simulator.

        \end{itemize}
    }
    
    %% Soft Skills Course
    \vspace{10pt}
    \simpleHeading{\textbf{Professional Soft Skills Coursework}}{\textit{USU Center for Anticipatory Intelligence}}

    \simpleParagraph{Acquired and practiced soft-skill tools and techniques in the areas of self-assessment, team
    facilitation, emotional intelligence, leadership, and creativity. Received training and practice
    from industry professionals and academics.}
    
%----------- PROJECTS -----------

    %% Potentially add Narrows if it becomes relevant
    %\vspace{10pt}
    %\simpleHeading{\textbf{Narrows - Cross Platform C/C++ Cross Thread Channels}}{\textit{Personal Project}}

    %\simpleParagraph{TODO.}
    
    %% TCP/IP Stack from networks
    %\vspace{10pt}
    %\simpleHeading{\textbf{WaddoupsNet - A TCP/IP Stack on Linux}}{\textit{USU ECE Networking Class Project}}

    %\simpleParagraph{TODO.}

%----------- SKILLS -----------

\section{Technical Skills}
  \begin{adjustwidth}{0.15in}{0.0in}
      \begin{center}
    \begin{tabular*}{0.8\linewidth}{@{\extracolsep{\fill}} cccc }
        C/C++
            & \href{https://dafny.org/}{Dafny}
            & Python
            & VHDL
            \vspace{5pt} \\
       
        Bash
            & \href{https://www.modestchecker.net/}{M\fakesc{odest} Toolset}
            & Rust
            & Linux
    \end{tabular*}
      \end{center}
  \end{adjustwidth}

%----------- VOLUNTEER WORK -----------

%\section{Interests}
%    \positionHeading{Service Volunteer}{Brisbane, Australia}
%        {Church of Jesus Christ of Latter-day Saints}{2018 -- 2020}
%
%    \positionHeading{Presidency Member}{Utah State University}
%        {Disc Golf Club}{2023 -- present}

%----------- CERTIFICATES -----------

%----------- ORGANIZATIONS -----------

%----------- HOBBIES -----------

%----------- REFERENCES -----------

\end{document}
